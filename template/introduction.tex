\chapter{序論}
\section{背景}
農業従事者の高齢化や後継者不足により,農作業の負担が増加している.特に生産の場ではない畦道での草刈り作業は小型機械などを用いた人力での作業が行われており,年に複数回行わなければならない,重労働で危険を伴う作業である.また農地内での荷物の運搬を人力で行う際には一輪車%画像差し込むべき?%
が使われており畦道上などの細い道では転倒するリスクがある.これらの問題のように農作業には従事者の負担や作業による危険性が多くあるため,負担などを減らすために作業の機械化が期待されている.そのなかで移動ロボットが畦道を走行することによって農作業の場での活躍が期待できると考えた.畦道を走行することで荷物の運搬や草刈り作業を自動化できることや,走行する際に畦道上の雑草に対して上から圧力を加えることで雑草の成長の抑制も期待できる[参考文献]など作業負担や作業に伴う危険性を減らすことが可能である.これらの例を実現するには移動ロボットが走行箇所である畦道を逸れることなく走行する必要があるが,畦道と水田には明確な境界がないために検出が困難なことや,傾斜から水田に滑り落ちることなどが考えられる.そのため移動ロボットが畦道を走行するためには畦道から逸れることなく,安全な走行が可能となる路面の検出が必要となる.
\section{従来研究}

%\ref{chap:purpose}%章で目的を述べる。

% dvipdfmxとhereのテスト
%\begin{figure}[H]
%	\begin{center}
%		\includegraphics[width=1.0\linewidth]{../zero.png}
%		\caption{}
%		\label{fig:}
%	\end{center}
%\end{figure}
%
