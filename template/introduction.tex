\chapter{序論}
移動ロボットが畦道を走行することにより農作業の場での様々な活躍が期待できる.例えば人力で行っている荷物の運搬作業を自動化することで運搬する人間の労力の軽減や,バランスを崩して転倒するリスクを削減できる.また,畦道での草刈り作業では小型機械などを用いた人力での作業が行われており重労働で危険を伴う作業であるが,移動ロボットに草刈りを行わせることや,走行する際に畦道上の雑草に対して上から圧力を加えることで雑草の成長の抑制が期待でき[参考文献],農業従事者の負担軽減につなげることができる.これらの例を実現するには移動ロボットが走行箇所である畦道を逸れることなく走行する必要があるが,畦道と水田には明確な境界がないために検出が困難なことや,傾斜から水田に滑り落ちることなどが考えられる.そのため移動ロボットが畦道を走行するためには畦道から逸れることなく,安全な走行が可能となる路面の検出が必要となる.

\ref{chap:purpose}章で目的を述べる。

% dvipdfmxとhereのテスト
%\begin{figure}[H]
%	\begin{center}
%		\includegraphics[width=1.0\linewidth]{../zero.png}
%		\caption{}
%		\label{fig:}
%	\end{center}
%\end{figure}
%
